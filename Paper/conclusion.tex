\section{Conclusions and Directions for Future Research} \label{sec:conclusion}

\subsection{Programming Problems}

Sketch synthesis and abstraction of Python programming problems has the potential to be a quick and easy way to automatically generate many problems for teachers and students. We have presented an overview of our pipeline and are working to coordinate these parts together to create a scalable system and a feasible user interface.

We currently have tested the different parts of the system independently of each other. We have also been able to generate AST�s for different programming problems, and translate simple ones into Sketch. We have been able to use Sketch separately from the other parts of the pipeline to generate AST�s, and we are currently working on putting the different parts together. 

In the future, we will be implementing the UI section of the system, which involves inserting holes into AST�s generated by Sketch, having a method for getting the users to fill in the blanks, determining if the user�s solution is correct, and providing feedback for what the user has done incorrectly and what they need to work on. We will also be making differents processes more automated, including abstracting with heuristics, generating the Sketch Specification, and generating optimal problems. The optimal problems that we want to automatically generate are not complex enough that the user cannot find a solution, but not so easy that the problem is no longer interesting to solve. 

\subsection{Math Problems}

