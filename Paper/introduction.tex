\section{Introduction} \label{sec:intro}

The large scale of students in popular courses have forced researches to develop new automated technologies to solve problems such as automated feedback generation \cite{autograder} and solution generation \cite{geometry}. Another important problem resulting from this scale is that of automated problem generation to cater to the practice needs of different students as well as for providing different exams to students of a given difficulty level. For example, massive online open courses (MOOCs) with thousands of students may constantly require new problems that are parameterized by complexity and by a set of concepts the students wish to practice.

In this work we present a constraint-based system to automatically generate problems that engage the user in practicing specific concepts. Furthermore, we demonstrate the practicality of this system by generating Python programming problems and math problems in two domains: analytic geometry and calculus derivatives. 

There has been some previous work on generating new problems in various domains, namely algebra and programming. The work on generating new algebra problems has mostly been looked upon using two main approaches. In the first approach, a teacher is provided a certain set of parameter values that are fixed for a given domain \cite{jurkovic}. For example, for generating a quadratic equation, the parameters can be the number of roots, difficulty of factorization, whether there is an imaginary root, the range of coefficient values etc. Given a set of feature valuations, the tool generates the corresponding quadratic equations. The second approach takes a particular proof problem, and tries to learn a problem template from the problem which is then instantiated with different concrete values \cite{algebra}. The system first tries to learn a general query from a given proof problem, which is then executed to generate a set of proof problems. Since the query is only a syntactic generalization of the original problem, only a subset of them are valid problems, which are identified using polynomial identity testing. 

More recently, a technique was proposed to generate fill-in-the-blank Java problems where certain keywords, variables and control symbols are removed randomly from a correct solution \cite{javablanks}. The technique blanks variables using the condition that at least one occurrence of each variable remains in the scope and blanks control symbols such that at least one occurrence of a paired symbol (such as brackets) remains. 

Our techniques for generating math and programming problems, however, are more general since they are constrained-based and can check for more interesting constraints such as unique solutions. Our system can also capture the notion of problem difficulty using complexity functions. 

The programming problems we generate take the form of partially complete programs with input/output examples. To solve the problem, the user must correctly fill in the blanks in the partial program based on the input/output constraints. Using this technique, we are able to generate simple Python problems that focus on a set of constructs, including arithmetic, lists and list operations, function calls, and control statements. We are working to extend our ADT grammar to support generation of more complex Python programming problems. We also aim to build a feasible user interface for presentation of this system.

The math problems we generate are specified by a specification class that defines the problem template, compute/correct methods, constraints, and bounds. An instance of this specification class is then passed into a solver. The solver then generates concrete parameters for the problem as constrained by the specification instance. Note that this technique separates the solver from parser, which allows us to try multiple different algorithms at each step. We are able to specify constraints that can generate problems of varying difficulty. Using this technique, we are able to generate standard textbook problems in analytic geometry and calculus derivatives. We are working to make the constraints be specified automatically rather than manually, which will allow the system to be more user-friendly for the general public.

In Section \ref{sec:programming}, we present our technique for generating Python programming problems. In Section \ref{sec:math}, we present our technique for generating math problems. In Section \ref{sec:conclusion}, we summarize our work and present directions for further research.

